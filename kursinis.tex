\documentclass{VUMIFPSkursinis}
\usepackage{algorithmicx}
\usepackage{algorithm}
\usepackage{algpseudocode}
\usepackage{amsfonts}
\usepackage{amsmath}
\usepackage{bm}
\usepackage{color}
\usepackage{hyperref}  % Nuorodų aktyvavimas
\usepackage{url}


% Titulinio aprašas
\university{Vilniaus universitetas}
\faculty{Matematikos ir informatikos fakultetas}
\department{Programų sistemų katedra}
\papertype{Kursinis darbas}
\title{Pakartotinis kodo panaudojimas pirminio kriptovaliutų platinimo (ICO) išmaniuosiuose kontraktuose}
\titleineng{Code review in initial coin offering (ICO) smart contracts}
\status{3 kurso 1 grupės studentė}
\author{Agnė Mačiukaitė}
\supervisor{lekt. Gediminas Rimša}
\date{Vilnius \\ \the\year}


\bibliography{library} 

\begin{document}
\maketitle

\tableofcontents


\sectionnonum{Įvadas}
Programinės įrangos pernaudojimas leidžia naudoti programas keliuose projektuose. Tai yra svarbi strategija programinei įrangai norint padidinti sistemos efektyvumą ir kokybę. Taikant kodo pernaudojimamą programuotojai naudojasi kodu, kurį keičia taip, kad jis atitiktu dabartinio projekto reikalavimus \cite {Ravichandran2003}.

Pirminio kriptovaliuto platinimo (angl. initial coin offering, toliau ICO) metu įmonė parduoda specializuotus kripto-žetonus žadėdami, kad žetonai veiks kaip mainų priemonė gaunat paslaugas įmonės platformoje. Žetonų pardavimas kuria kapitalą pradiniam įmonės platformos kūrimui nors nėra įsipareigojimo dėl būsimos paslaugos kainos (žetonais ar kitaip) \cite{Catalini2018}. Satoshi Nakamoto išleidęs baltąjį popierių (angl. whitepaper) \cite{Nakamoto2008} įvykdė ICO  ir taip surinko finansavimą pirmąjam blockchain ir kriptovaliutai Bitcoin. Bitcoin - skaitmeniniai pinigai, kurių pavedimai vyksta internete naudojantis decentralizuota vieša duomenų baze - blockchain \cite{Swan2015}. Šiuo metu du populiariausi blockchain yra Ethereum ir Bitcoin \cite{Luu}. Ethereum be savo kriptovaliutos turi ir kitą svarbų funkcionalumą - išmaniuosius kontraktus - Turing complete programą, kuri leidžia rašyti decentralizuotas aplikacijas \cite{Buterin2014}. Solidity - populiariausia kalba naudojama rašyti išmaniesiems kontraktams \cite{Dannen}. Problema - išmaniųjų kontraktų technologijos yra pakankamai jaunos, dėl to pakartotonio kodo panaudojimo bazė dar tik formuojasi. ICO kontraktai yra tiražuojami kopijavimo su modifikacijos būdu.

Programinės įrangos produktų linija (angl. product line software engineering, toliau PLSE) naudojama įmonėse pakartojamumui susijusiuose programinės įrangos produktuose numatyti. PLSE suteikia bendrą architektūrą ir pernaudojamą kodą programinės įrangos kūrėjams \cite{Svahnberg}. Toks kūrimas susideda iš savybių išskyrimo ir jų įgyvendinimo produkte. Gerai išskirtos produkto ypatybės padeda sukurti lengvai pernaudojamą programą. Savybės turi būti atrinktos atsižvelginat į jų paplitimą bei kintamumą srityje \cite{Lee2015}. Naudojantis PLSE produkto kūrėjai gali fokusuotis produkto specifikacijoje, o ne bendrų savybėse \cite{Svahnberg}.

Savybių modeliavimas yra pagrindinis metodas atrinkti bei valdyti bendrąsias ir kintamas savybes produktų linijoje. Programinės įrangos šeimos gyvavimo pradžioje savybių modelis padeda išskirti pagrindines savybes, kurios gelbsti kuriant naują rinką ar  norint išlikti jau esamoje. Taip pat savybių modelis leidžia išskirti rizikingas savybes, nuspėti, kokia yra visos programos ar atskirų savybių kaina. Vėliau savybių modeliavimas padeda išskirti variacijos taškus programinės įrangos architektūroje \cite{Czarnecki2004}. Savybių modeliavimas yra populiariausias PLSE kūrime nuo pat pirmojo jo pristatymo \cite{Kang1990}. Taip yra todėl, nes savybės yra pakankamai abstraktus konseptas padedantis efektyviai bendrauti suinterasuotoms šalims. Savybių modeliavimas yra intuitivus ir efektyvus būdas žmonėms išreikšti savybių paplitimą ir kintamumą programinės įrangos šeimoje \cite{Kang2013}. 

Šio darbo tikslas - ištirti pirminio finansavimo kriptovaliutomis (ICO) išmaniuosius kontraktus, nustatyti, kokios savybės yra pastavios, o kokios - kintamos bei pasiūlyti būdus kodo pernaudojamumui didinti. 

Tikslui pasiekti išsikelti uždaviniai:
\begin{enumerate}
\item Apžvelgti savybių modeliavimą programinės įrangos produktų linijos sričiai 
\item Surinkti virš 100 išmaniųjų kontraktų skirtų ICO
\item Išskirti surinktų kontraktų savybes į pastovias ir kintančias
\item Pasiūlyti ICO išmaniuosius kontraktus pagal išrinktas savybes
\end{enumerate}

\section{Savybių modeliavimas}
Savybių modeliavime bendri ir kintami bruožai yra modeliuojami iš produkto savybių perspektyvos PLSE, kuri yra suinterestuotos šalies interesas. Originalus savybių modeliavimas - FODA \cite{Kang1990} - paprastas modelis, kuris savybes skirsto pagal tai iš ko jos susideda bei pagal bendrumą ir specializaciją naudojant AND/OR  diagramas. Savybės yra suskirstytos į būtinas, alternatyvias ir pasirenkamas pagal bendrus ir kintamus bruožus. Savybių atributai taip pat gali būti dokumentuojami \cite{Kang2013}.

\subsection{Savybė}

Savybės yra pagrindinis produkto skiriamasis bruožas. Skirtingi srities analizės metodai terminą „savybė" apibūdina šiek tiek kitaip. FODA \cite{Kang1990} savybę apibūdina kaip pastebimą ir skiriamą sistemos charakteristiką, kuri yra matoma įvairioms suinteresuotoms šalims. Svarbu, kad savybių modeliavime turi būti fokusuojamasi ties bendrumo ir skirtumų srityje identifikacija, o ne ties bendrų savybių supratimo išskyrimu. Iš kitos pusės neapibrėžta savybių paskirtis daro sunkumų formuluojant jos semantiką, rezultatų valdymą bei automatinį pagalbo suteikimą.

Skirtumas tarp savybės ir konceptualios abstrakcijos (pvz.: funkcijos, objekto) yra tai, kad funkcijos ir objektai yra naudojami specifikuojant vidines sistemos detales. Kitaip, funkcijos ir objektai yra konceptualios abstrakcijos, kurios yra identifikuojamos iš vidinės sistemos pusės. Savybė - aiškiai matoma  pagal charakteristiką, kuri gali išskirti produktą iš kitų. Todėl savybių modeliavimas turi išskirti iš išorės matomas charakteristikas produktuose bendrumo ir kintamumo atžvilgiu, o ne apibūdinti visas produkto modeliavimo detales (pvz.: funkcinis, objektais orientuotas modeliavimas). Suprantant produkto bendrus ir kintamus bruožus galima sukurti pernaudojamas funkcijas ir objektus \cite{Lee2015}.

Savybėmis grįsti PLSE modeliai naudoja savybes kaip vienetus sudarytus iš:
\begin{itemize}
\item Veiksmų, kurie suteikiami vartotojams
\item Reikalavimų
\item Produkto konfigūracijos ir jos valdymo
\item Kūrimo ir pristatymo klientams 
\item Parementrizacijos pernaudojamies vienetams
\item Produkto valdymo skirtingiems rinkos sektoriams \cite{Kang2013}
\end{itemize}
\subsection{Savybių modelis}
\section{Savybių modeliavimas pirminio kriptovaliutų platinimo išmaniesiems kontraktams}
\subsection{Savybė}
\subsection{Savybių modeliavimas}


\sectionnonum{Rezultatai}



\sectionnonum{Išvados}
Išvadose ir pasiūlymuose, nekartojant atskirų dalių apibendrinimų,
suformuluojamos svarbiausios darbo išvados, rekomendacijos bei pasiūlymai.



\printbibliography[heading=bibintoc] % Literatūros šaltiniai aprašomi
%bibliografija.bib faile. Šaltinių sąraše nurodoma panaudota literatūra,
%kitokie šaltiniai. Abėcėlės tvarka išdėstoma tik darbe panaudotų (cituotų,
%perfrazuotų ar bent paminėtų) mokslo leidinių, kitokių publikacijų
%bibliografiniai aprašai (šiuo punktu pasirūpina LaTeX). Aprašai pateikiami
% netransliteruoti.

\sectionnonum{Sąvokų apibrėžimai}
Turing complete programa - sistema, kuri yra pakankamai galinga atpažinti visus galimus algoritmus \cite{Teller1994}. 

\sectionnonum{Santrumpos}
PLSE - programinės įrangos produktų linija (angl. product line software engineering)

ICO - pirminis kriptovaliutų platinimas (angl. initial coin offering)

%\appendix  % Priedai
% Prieduose gali būti pateikiama pagalbinė, ypač darbo autoriaus savarankiškai
% parengta, medžiaga. Savarankiški priedai gali būti pateikiami kompiuterio
% diskelyje ar kompaktiniame diske. Priedai taip pat vadinami ir numeruojami.
% Tekstas su priedais siejamas nuorodomis (pvz.: \ref{img:mlp}).
%

\end{document}
