\documentclass{VUMIFInfKursinis}
\usepackage{algorithmicx}
\usepackage{algorithm}
\usepackage{algpseudocode}
\usepackage{amsfonts}
\usepackage{amsmath}
\usepackage{bm}
\usepackage{color}
% \usepackage{hyperref}  % Nuorodų aktyvavimas
\usepackage{url}


% Titulinio aprašas
\university{Vilniaus universitetas}
\faculty{Matematikos ir informatikos fakultetas}
\department{Programų sistemų katedra}
\papertype{Kursinis darbas}
\title{Pakartotinis kodo panaudojimas pirminio kriptovaliutų platinimo (ICO) išmaniuosiuose kontraktuose}
\titleineng{Code review in initial coin offering (ICO) smart contracts}
\status{3 kurso 1 grupės studentė}
\author{Agnė Mačiukaitė}
\supervisor{lekt. Gediminas Rimša}
\date{Vilnius \\ \the\year}

% Nustatymai
% \setmainfont{Palemonas}   % Pakeisti teksto šriftą į Palemonas (turi būti įdiegtas sistemoje)
\bibliography{library} 
%
\begin{document}
\maketitle

\tableofcontents

%\sectionnonum{Sąvokų apibrėžimai}
%Sutartinių ženklų, simbolių, vienetų ir terminų sutrumpinimų sąrašas (jeigu
%ženklų, simbolių, vienetų ir terminų bendras skaičius didesnis nei 10 ir
%kiekvienas iš jų tekste kartojasi daugiau nei 3 kartus).
%
\sectionnonum{Įvadas}

Tyrimo problema: Išmaniųjų kontraktų technologijos yra pakankamai jaunos, dėl to projektavimo šablonai bei pakartotinai panaudojamo kodo bazė dar tik formuojasi. ICO kontraktai tiražuojami kopijavimo su modifikacijomis būdu.

Tyrimo tikslas: Ištirti sutelktinio finansavimo kriptovaliutomis (ICO) išmaniuosius kontraktus, nustatyti koks funkcionalumas yra pastovus, o koks - kintantis, bei pasiūlyti būdus pakartotinio panaudojamumo laipsniui didinti.
Įvade apibūdinamas darbo tikslas, temos aktualumas ir siekiami rezultatai.

\section{Kintamumo modeliavimas (Variability modeling)}

\subsection{Savybių modeliavimas (Feature modeling)}
Citavimo pavyzdžiai: cituojamas vienas šaltinis \cite{Berger2010};
\subsubsection{Savybė}
\subsubsection{Savybių modelis}
\subsubsection{Savybių modeliavimo sistemos}
\subsection{Sprendimų modeliavimas (Decision modeling)}
\section{Programinės įrangos produktų linija}
\subsection{Produktų linijos savybių modeliavimas}
\subsection{Savybėmis orientuotas programavimas}
\section{Blockchain}
\subsection{Blockchain 1.0}
\subsubsection{Kriptovaliutos}
\subsection{Blockchain 2.0}
\subsubsection{Išmanusis kontraktas}
\subsubsection{Pirminis kriptovaliutų platinimas (ICO)}
\subsubsection{Ethereum}
\section{Pirminio kriptovaliuto platinimo išmaniojo kontrakto savybių modeliavimas}
\subsection{Savybės}
\subsection{Savybių modelis}
\subsection{Savybėmis grįstas išmanusis kontraktas pirminiui kriptovaliutų platinimui}



\sectionnonum{Rezultatai}



\sectionnonum{Išvados}
Išvadose ir pasiūlymuose, nekartojant atskirų dalių apibendrinimų,
suformuluojamos svarbiausios darbo išvados, rekomendacijos bei pasiūlymai.



\printbibliography[heading=bibintoc] % Literatūros šaltiniai aprašomi
%bibliografija.bib faile. Šaltinių sąraše nurodoma panaudota literatūra,
%kitokie šaltiniai. Abėcėlės tvarka išdėstoma tik darbe panaudotų (cituotų,
%perfrazuotų ar bent paminėtų) mokslo leidinių, kitokių publikacijų
%bibliografiniai aprašai (šiuo punktu pasirūpina LaTeX). Aprašai pateikiami
% netransliteruoti.

\appendix  % Priedai
% Prieduose gali būti pateikiama pagalbinė, ypač darbo autoriaus savarankiškai
% parengta, medžiaga. Savarankiški priedai gali būti pateikiami kompiuterio
% diskelyje ar kompaktiniame diske. Priedai taip pat vadinami ir numeruojami.
% Tekstas su priedais siejamas nuorodomis (pvz.: \ref{img:mlp}).
%

\end{document}
