\documentclass{VUMIFInfKursinis}
\usepackage{algorithmicx}
\usepackage{algorithm}
\usepackage{algpseudocode}
\usepackage{amsfonts}
\usepackage{amsmath}
\usepackage{bm}
\usepackage{color}
\usepackage{hyperref}  % Nuorodų aktyvavimas
\usepackage{url}


% Titulinio aprašas
\university{Vilniaus universitetas}
\faculty{Matematikos ir informatikos fakultetas}
\department{Programų sistemų katedra}
\papertype{Kursinis darbas}
\title{Pakartotinis kodo panaudojimas pirminio kriptovaliutų platinimo (ICO) išmaniuosiuose kontraktuose}
\titleineng{Code review in initial coin offering (ICO) smart contracts}
\status{3 kurso 1 grupės studentė}
\author{Agnė Mačiukaitė}
\supervisor{lekt. Gediminas Rimša}
\date{Vilnius \\ \the\year}


\bibliography{library} 

\begin{document}
\maketitle

\tableofcontents

%\sectionnonum{Sąvokų apibrėžimai}
%Sutartinių ženklų, simbolių, vienetų ir terminų sutrumpinimų sąrašas (jeigu
%ženklų, simbolių, vienetų ir terminų bendras skaičius didesnis nei 10 ir
%kiekvienas iš jų tekste kartojasi daugiau nei 3 kartus).
%
\sectionnonum{Įvadas}
Programinės įrangos produktų linija (angl. product line software engineering, toliau PLSE) yra nauja paradigma programų kūrime, kuri kuria programas naudodamasi jau žinomis esminėmis savybės, taip išvengdama produktų kūrimo nuo nulio. Toks kūrimas susideda iš savybių išskyrimo ir jų įgyvendinimo produkte. Gerai išskirtos produkto ypatybės padeda sukurti lengvai pernaudojamą programą. Savybės turi būti atrinktos atsižvelginat į jų paplitimą bei kintamumą srityje. Produktų linijos susiaurinimas iki srities padidina pernaudojamą bei panaudojamumą \cite{Lee2015}. \textbf{PABAIGTI}

Savybių modeliavimas yra pagrindinis metodas atrinkti bei valdyti bendrąsias ir kintamas savybes produktų linijoje. Programinės įrangos šeimos gyvavimo pradžioje savybių modelis padeda išskirti pagrindines savybes, kurios gelbsti kuriant naują rinką ar  norint išlikti jau esamoje. Taip pat savybių modelis leidžia išskirti rizikingas savybes, nuspėti, kokia yra visos programos ar atskirų savybių kaina. Vėliau savybių modeliavimas padeda išskirti variacijos taškus programinės įrangos architektūroje \cite{Czarnecki2004}. Savybių modeliavimas yra populiariausias PLSE kūrime nuo pat pirmojo jo pristatymo \cite{Kang1990}. Taip yra todėl, nes savybės yra pakankamai abstraktus konseptas padedantis efektyviai bendrauti suinterasuotoms šalims. Savybių modeliavimas yra intuitivus ir efektyvus būdas žmonėms išreikšti savybių paplitimą ir kintamumą programinės įrangos šeimoje \cite{Kang2013}. 

Blockchain technologija yra jauna programinės įrangos produktų linija, kuri atsirado tik 2009, kai Satoshi Nakamoto išleidęs baltąjį popierių (angl. whitepaper) \cite{Nakamoto2008} įvykdė pirminį kriptovaliutų platinimą (angl. initial coin offering, toliau ICO)  ir taip surinko finansavimą pirmąjam blockchain ir kriptovaliutai Bitcoin. Bitcoin - skaitmeniniai pinigai, kurių pavedimai vyksta internete naudojantis decentralizuota vieša duomenų baze - blockchain \cite{Swan2015}. Šiuo metu du populiariausi blockchain yra Ethereum ir Bitcoin \cite{Luu}. Ethereum be savo kriptovaliutos turi ir kitą svarbų funkcionalumą - išmaniuosius kontraktus - Turing complete programą, kuri leidžia rašyti decentralizuotas aplikacijas \cite{Buterin2014}. Solidity - populiariausia kalba naudojama rašyti išmaniesiems kontraktams \cite{Dannen}. Problema - išmaniųjų kontraktų technologijos yra pakankamai jaunos, dėl to pakartotonio kodo panaudojimo bazė dar tik formuojasi. ICO kontraktai yra tiražuojami kopijavimo su modifikacijos būdu.

Šio darbo tikslas - ištirti pirminio finansavimo kriptovaliutomis (ICO) išmaniuosius kontraktus, nustatyti, kokios savybės yra pastavios, o kokios - kintamos bei pasiūlyti būdus kodo pernaudojamumui didinti. 

Tikslui pasiekti išsikelti uždaviniai:
\begin{enumerate}
\item Apžvelgti savybių modeliavimą programinės įrangos produktų linijos sričiai 
\item Surinkti virš 100 išmaniųjų kontraktų skirtų ICO
\item Išskirti surinktų kontraktų savybes į pastovias ir kintančias
\item Pasiūlyti ICO išmaniuosius kontraktus pagal išrinktas savybes
\end{enumerate}

\section{Savybių modeliavimas programinės įrangos produktų linijos sričiai}
\section{Savybių modeliavimas ICO išmaniesiems kontraktams}


\sectionnonum{Rezultatai}



\sectionnonum{Išvados}
Išvadose ir pasiūlymuose, nekartojant atskirų dalių apibendrinimų,
suformuluojamos svarbiausios darbo išvados, rekomendacijos bei pasiūlymai.



\printbibliography[heading=bibintoc] % Literatūros šaltiniai aprašomi
%bibliografija.bib faile. Šaltinių sąraše nurodoma panaudota literatūra,
%kitokie šaltiniai. Abėcėlės tvarka išdėstoma tik darbe panaudotų (cituotų,
%perfrazuotų ar bent paminėtų) mokslo leidinių, kitokių publikacijų
%bibliografiniai aprašai (šiuo punktu pasirūpina LaTeX). Aprašai pateikiami
% netransliteruoti.

\sectionnonum{Sąvokų apibrėžimai}
Turing complete programa - sistema, kuri yra pakankamai galinga atpažinti visus galimus algoritmus \cite{Teller1994}. 

\sectionnonum{Santrumpos}
PLSE - programinės įrangos produktų linija (angl. product line software engineering).

ICO - pirminis kriptovaliutų platinimas (angl. initial coin offering).

\appendix  % Priedai
% Prieduose gali būti pateikiama pagalbinė, ypač darbo autoriaus savarankiškai
% parengta, medžiaga. Savarankiški priedai gali būti pateikiami kompiuterio
% diskelyje ar kompaktiniame diske. Priedai taip pat vadinami ir numeruojami.
% Tekstas su priedais siejamas nuorodomis (pvz.: \ref{img:mlp}).
%

\end{document}
