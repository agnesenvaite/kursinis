\documentclass{VUMIFInfKursinis}
\usepackage{algorithmicx}
\usepackage{algorithm}
\usepackage{algpseudocode}
\usepackage{amsfonts}
\usepackage{amsmath}
\usepackage{bm}
\usepackage{color}
% \usepackage{hyperref}  % Nuorodų aktyvavimas
\usepackage{url}


% Titulinio aprašas
\university{Vilniaus universitetas}
\faculty{Matematikos ir informatikos fakultetas}
\department{Programų sistemų katedra}
\papertype{Kursinis darbas}
\title{Pakartotinis kodo panaudojimas pirminio kriptovaliutų platinimo (ICO) išmaniuosiuose kontraktuose}
\titleineng{Code review in initial coin offering (ICO) smart contracts}
\status{3 kurso 1 grupės studentė}
\author{Agnė Mačiukaitė}
\supervisor{lekt. Gediminas Rimša}
\date{Vilnius \\ \the\year}

% Nustatymai
% \setmainfont{Palemonas}   % Pakeisti teksto šriftą į Palemonas (turi būti įdiegtas sistemoje)
\bibliography{library} 
%
\begin{document}
\maketitle

\tableofcontents

%\sectionnonum{Sąvokų apibrėžimai}
%Sutartinių ženklų, simbolių, vienetų ir terminų sutrumpinimų sąrašas (jeigu
%ženklų, simbolių, vienetų ir terminų bendras skaičius didesnis nei 10 ir
%kiekvienas iš jų tekste kartojasi daugiau nei 3 kartus).
%
\sectionnonum{Įvadas}
Programinės įrangos produktų linijos paradigma vadovaujasi tuo, kad produktus reikia kuri iš esminių savybių, o kiekvieno produkto kūrimas nuo nulio \cite{Lee2015}. Tinkamų savybių išskyrimas yra labai svarbus norint sukurti produktą, kuris pasižymi geromis pernaudojimo savybėmis. Geras produktų linijos projektavimas pasižymi ne tik bendrų aspektų išskyrimu ar kintamumo modeliavimu, bet ir srities išmanymu, kuris duoda platesnę projektavimo perspektyvą ir pritaikomumą \cite{Lee2015}.

Savybė

Savybių modeliavimas yra pagrindinis metodas nustatyti ir valdyti bendras ir kintamas savybės sistemose, sistemų šeimose ar produktų linijoje. Ankstyvame produktų linijos kūrime savybių modeliai apibūdina sritį, registruoja ir vertina  reikalingą informaciją. Vėliau savynių modeliai užima svarbią vietą sistemos šeimos architektūroje, kuri turi suprasti pagrindinius variacijos tikslus \cite{ĮRAŠYTI}.




Pirmasis savybių modeliavimas, FODA \cite{ĮRAŠYTI ŠALTINĮ}, yra paprastas modelis, kuris savybes organizuoja naudodamas "susidaro iš" ir "apibendrinimu/specifikavimu" santykius naudojantis IR/AR grafu. Savybės yra suskirstytos į būtinas, arternatyvias ir pasirinktinas taip atvaizduojant bendrumą ir kintamumą \cite{Kang2013}. FODA yra originalus savybių modeliavimo būdas, tačiau po jo išleidimo sekė ir kiti savybių modeliavimo būdai. 
 
Blockchain kaip programinės įrangos šeima yra labai jauna. Nors technologija buvo aprašyta dar praeitame amžiuje \cite{ĮRAŠYTI ŠALTINius}. Pirmą kartą ji įgyvendinta tik 2009 metais Satoshi Nokamoto. 

----------------------------------------------------------------------------------

Tyrimo problema: Išmaniųjų kontraktų technologijos yra pakankamai jaunos, dėl to projektavimo šablonai bei pakartotinai panaudojamo kodo bazė dar tik formuojasi. ICO kontraktai tiražuojami kopijavimo su modifikacijomis būdu.

Tyrimo tikslas: Ištirti sutelktinio finansavimo kriptovaliutomis (ICO) išmaniuosius kontraktus, nustatyti koks funkcionalumas yra pastovus, o koks - kintantis, bei pasiūlyti būdus pakartotinio panaudojamumo laipsniui didinti.
Įvade apibūdinamas darbo tikslas, temos aktualumas ir siekiami rezultatai.

\section{Kintamumo modeliavimas (Variability modeling)}

\subsection{Savybių modeliavimas (Feature modeling)}
Citavimo pavyzdžiai: cituojamas vienas šaltinis \cite{Berger2010};
\subsubsection{Savybė}
\subsubsection{Savybių modelis}
\subsubsection{Savybių modeliavimo sistemos}
\subsection{Sprendimų modeliavimas (Decision modeling)}
\section{Programinės įrangos produktų linija}
\subsection{Produktų linijos savybių modeliavimas}
\subsection{Savybėmis orientuotas programavimas}
\section{Blockchain}
\subsection{Blockchain 1.0}
\subsubsection{Kriptovaliutos}
\subsection{Blockchain 2.0}
\subsubsection{Išmanusis kontraktas}
\subsubsection{Pirminis kriptovaliutų platinimas (ICO)}
\subsubsection{Ethereum}
\section{Pirminio kriptovaliuto platinimo išmaniojo kontrakto savybių modeliavimas}
\subsection{Savybės}
\subsection{Savybių modelis}
\subsection{Savybėmis grįstas išmanusis kontraktas pirminiui kriptovaliutų platinimui}



\sectionnonum{Rezultatai}



\sectionnonum{Išvados}
Išvadose ir pasiūlymuose, nekartojant atskirų dalių apibendrinimų,
suformuluojamos svarbiausios darbo išvados, rekomendacijos bei pasiūlymai.



\printbibliography[heading=bibintoc] % Literatūros šaltiniai aprašomi
%bibliografija.bib faile. Šaltinių sąraše nurodoma panaudota literatūra,
%kitokie šaltiniai. Abėcėlės tvarka išdėstoma tik darbe panaudotų (cituotų,
%perfrazuotų ar bent paminėtų) mokslo leidinių, kitokių publikacijų
%bibliografiniai aprašai (šiuo punktu pasirūpina LaTeX). Aprašai pateikiami
% netransliteruoti.

\appendix  % Priedai
% Prieduose gali būti pateikiama pagalbinė, ypač darbo autoriaus savarankiškai
% parengta, medžiaga. Savarankiški priedai gali būti pateikiami kompiuterio
% diskelyje ar kompaktiniame diske. Priedai taip pat vadinami ir numeruojami.
% Tekstas su priedais siejamas nuorodomis (pvz.: \ref{img:mlp}).
%

\end{document}
