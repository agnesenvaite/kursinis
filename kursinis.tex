\documentclass{VUMIFInfKursinis}
\usepackage{algorithmicx}
\usepackage{algorithm}
\usepackage{algpseudocode}
\usepackage{amsfonts}
\usepackage{amsmath}
\usepackage{bm}
\usepackage{color}
% \usepackage{hyperref}  % Nuorodų aktyvavimas
\usepackage{url}


% Titulinio aprašas
\university{Vilniaus universitetas}
\faculty{Matematikos ir informatikos fakultetas}
\department{Programų sistemų katedra}
\papertype{Kursinis darbas}
\title{Pakartotinis kodo panaudojimas pirminio kriptovaliutų platinimo (ICO) išmaniuosiuose kontraktuose}
\titleineng{Code review in initial coin offering (ICO) smart contracts}
\status{3 kurso 1 grupės studentė}
\author{Agnė Mačiukaitė}
\supervisor{lekt. Gediminas Rimša}
\date{Vilnius \\ \the\year}


\bibliography{library} 

\begin{document}
\maketitle

\tableofcontents

%\sectionnonum{Sąvokų apibrėžimai}
%Sutartinių ženklų, simbolių, vienetų ir terminų sutrumpinimų sąrašas (jeigu
%ženklų, simbolių, vienetų ir terminų bendras skaičius didesnis nei 10 ir
%kiekvienas iš jų tekste kartojasi daugiau nei 3 kartus).
%
\sectionnonum{Įvadas}
Programinės įrangos produktų linija yra nauja paradigma programų kūrime, kuri kuria programas naudodamasi jau žinomis esminėmis savybės, taip išvengdama produktų kūrimo nuo nulio. Toks kūrimas susideda iš savybių išskyrimo ir jų įgyvendinimo produkte. Gerai išskirtos produkto ypatybės padeda sukurti lengvai pernaudojamą programą. Savybės turi būti atrinktos atsižvelginat į jų paplitimą bei kintamumą srityje. Produktų linijos susiaurinimas iki srities padidina pernaudojamą bei panaudojamumą \cite{Lee2015}. \textbf{PABAIGTI}

Savybių modeliavimas yra pagrindinis metodas atrinkti bei valdyti bendrąsias ir kintamas savybes produktų linijoje. Programinės įrangos šeimos gyvavimo pradžioje savybių modelis padeda išskirti pagrindines savybes, kurios gelbsti kuriant naują rinką ar  norint išlikti jau esamoje. Taip pat savybių modelis leidžia išskirti rizikingas savybes, nuspėti, kokia yra visos programos ar atskirų savybių kaina. Vėliau savybių modeliavimas padeda išskirti variacijos taškus programinės įrangos architektūroje \cite{Czarnecki2004}. Savybių modeliavimas yra populiariausias programinės įrangos produktų linijų kurime nuo pat pirmojo jo pristatymo \cite{Kang1990}. Taip yra todėl, nes savybės yra pakankamai abstraktus konseptas padedantis efektyviai bendrauti suinterasuotoms šalims. Savybių modeliavimas yra intuitivus ir efektyvus būdas žmonėms išreikšti savybių paplitimą ir kintamumą programinės įrangos šeimoje \cite{Kang2013}.   
 

----------------------------------------------------------------------------------

Tyrimo problema: Išmaniųjų kontraktų technologijos yra pakankamai jaunos, dėl to projektavimo šablonai bei pakartotinai panaudojamo kodo bazė dar tik formuojasi. ICO kontraktai tiražuojami kopijavimo su modifikacijomis būdu.

Tyrimo tikslas: Ištirti sutelktinio finansavimo kriptovaliutomis (ICO) išmaniuosius kontraktus, nustatyti koks funkcionalumas yra pastovus, o koks - kintantis, bei pasiūlyti būdus pakartotinio panaudojamumo laipsniui didinti.
Įvade apibūdinamas darbo tikslas, temos aktualumas ir siekiami rezultatai.

\section{Kintamumo modeliavimas (Variability modeling)}

\subsection{Savybių modeliavimas (Feature modeling)}
Citavimo pavyzdžiai: cituojamas vienas šaltinis \cite{Batory2005};

\subsubsection{Savybė}
\subsubsection{Savybių modelis}
\subsubsection{Savybių modeliavimo sistemos}
\subsection{Sprendimų modeliavimas (Decision modeling)}
\section{Programinės įrangos produktų linija}
\subsection{Produktų linijos savybių modeliavimas}
\subsection{Savybėmis orientuotas programavimas}
\section{Blockchain}
\subsection{Blockchain 1.0}
\subsubsection{Kriptovaliutos}
\subsection{Blockchain 2.0}
\subsubsection{Išmanusis kontraktas}
\subsubsection{Pirminis kriptovaliutų platinimas (ICO)}
\subsubsection{Ethereum}
\section{Pirminio kriptovaliuto platinimo išmaniojo kontrakto savybių modeliavimas}
\subsection{Savybės}
\subsection{Savybių modelis}
\subsection{Savybėmis grįstas išmanusis kontraktas pirminiui kriptovaliutų platinimui}



\sectionnonum{Rezultatai}



\sectionnonum{Išvados}
Išvadose ir pasiūlymuose, nekartojant atskirų dalių apibendrinimų,
suformuluojamos svarbiausios darbo išvados, rekomendacijos bei pasiūlymai.



\printbibliography[heading=bibintoc] % Literatūros šaltiniai aprašomi
%bibliografija.bib faile. Šaltinių sąraše nurodoma panaudota literatūra,
%kitokie šaltiniai. Abėcėlės tvarka išdėstoma tik darbe panaudotų (cituotų,
%perfrazuotų ar bent paminėtų) mokslo leidinių, kitokių publikacijų
%bibliografiniai aprašai (šiuo punktu pasirūpina LaTeX). Aprašai pateikiami
% netransliteruoti.

\appendix  % Priedai
% Prieduose gali būti pateikiama pagalbinė, ypač darbo autoriaus savarankiškai
% parengta, medžiaga. Savarankiški priedai gali būti pateikiami kompiuterio
% diskelyje ar kompaktiniame diske. Priedai taip pat vadinami ir numeruojami.
% Tekstas su priedais siejamas nuorodomis (pvz.: \ref{img:mlp}).
%

\end{document}
